%%%%%%%%%%%%%%%%%%%%%%%%%%%%%%%%%%%%%%%%%%%%%%%%%%%%%%%%%%%%%%%%%%%%%%%%%%%%%%%%
%                               RESPONSE TO REVIEWERS
%%%%%%%%%%%%%%%%%%%%%%%%%%%%%%%%%%%%%%%%%%%%%%%%%%%%%%%%%%%%%%%%%%%%%%%%%%%%%%%%

\pagebreak
\newpage
\renewcommand*{\thesection}{\Alph{section}}
\nobalance
\section*{RESPONSE TO REVIEWERS}
\section*{Comments from the editor}

\review{I would like to thank the authors for submitting their work to TSE. While all the reviewers agree that the problem addressed is important and the paper has potential, the reviewers also identified several concerns that need to be carefully addressed. The reviewers also agreed that the amount of work needed to address those concerns is beyond a major revision. Thus, after carefully checking the reviewers and in accordance with reviewer’s recommendations, I also recommend “Revise and resubmit as New”.}

Thank you very much for your encouraging comments. We have made appropriate changes to the manuscript taking into careful consideration the recommendations of the reviewers. We hope that this new draft adequately satisfies all the issues raised by the reviewers. To assist the reviewers in tracking all the new changes, where appropriate, we have prefixed the text with \respto{X-XX} to correspond to the reviewers' questions.

Below is the brief summary of our changes:
\be
\item The problem statement needs to be clarified and properly positioned (R1, R3). XXX

\item The novelty of the approach needs to be clarified (R3). XXX (see \bareresp{2-1})

\item The writing needs to be improved (R1, R2, R3).

\item The study needs to include all the details to ensure replicability (R1, R2).

\item The results of the study need to be revisited since the reviewers are not convinced that the results support the claims in the paper (R1, R2, R3).

\item The background on transfer learning needs citations (R2).
\ee


\section*{Reviewer 1}

Thank you very much for your detailed review. We have made some additional corrections, clarifications, and revisions, as directed by your comments. To assist in your review, we refer to specific locations of each of the changes using \respto{1-XX}.

\review{(1.1) 1. Background. The paper builds upon a number of previous studies. However, several key questions have no answer: *What is* the bellwhether? *Why is* this relevant? How does it work when applied to different contexts? Why is this useful in practice? None of these questions are addressed in the paper. This submission should be self-contained and allow a reader to understand its content without reading 10 other papers; in this sense, the submission fails completely. For example, the paper mentions that: "The core intuition of this new approach is that if many projects are similar, then we do not need to run comparisons between all pairs of projects": very good, but where this intuition come from?}


